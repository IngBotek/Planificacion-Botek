\documentclass[11pt]{charter}

% El títulos de la memoria, se usa en la carátula y se puede usar el cualquier lugar del documento con el comando \ttitle
\titulo{Solución georeferenciada de monitoreo de ganado} 

% Nombre del posgrado, se usa en la carátula y se puede usar el cualquier lugar del documento con el comando \degreename
%\posgrado{Carrera de Especialización en Sistemas Embebidos} 
\posgrado{Carrera de Especialización en Internet de las Cosas} 
%\posgrado{Carrera de Especialización en Intelegencia Artificial}
%\posgrado{Maestría en Sistemas Embebidos} 
%\posgrado{Maestría en Internet de las cosas}

% Tu nombre, se puede usar el cualquier lugar del documento con el comando \authorname
\autor{Christian Canaan Castro Botek} 

% El nombre del director y co-director, se puede usar el cualquier lugar del documento con el comando \supname y \cosupname y \pertesupname y \pertecosupname
\director{Nombre del Director}
\pertenenciaDirector{pertenencia} 
% FIXME:NO IMPLEMENTADO EL CODIRECTOR ni su pertenencia
\codirector{} % si queda vacio no se deberíá incluir 
\pertenenciaCoDirector{}

% Nombre del cliente, quien va a aprobar los resultados del proyecto, se puede usar con el comando \clientename y \empclientename
\cliente{Mg. Ing. Osvaldo P. Ivani}
\empresaCliente{Smartium}

% Nombre y pertenencia de los jurados, se pueden usar el cualquier lugar del documento con el comando \jurunoname, \jurdosname y \jurtresname y \perteunoname, \pertedosname y \pertetresname.
\juradoUno{Nombre y Apellido (1)}
\pertenenciaJurUno{pertenencia (1)} 
\juradoDos{Nombre y Apellido (2)}
\pertenenciaJurDos{pertenencia (2)}
\juradoTres{Nombre y Apellido (3)}
\pertenenciaJurTres{pertenencia (3)}
 
\fechaINICIO{25 de Agosto de 2020}		%Fecha de inicio de la cursada de GdP \fechaInicioName
\fechaFINALPlanificacion{29 de Septiembre de 2020} 	%Fecha de final de cursada de GdP
\fechaFINALTrabajo{20 de Julio de 2021}		%Fecha de defensa pública del trabajo final


\begin{document}

\maketitle
\thispagestyle{empty}
\pagebreak


\thispagestyle{empty}
{\setlength{\parskip}{0pt}
\tableofcontents{}
}
\pagebreak


\section{Registros de cambios}
\label{sec:registro}


\begin{table}[ht]
\label{tab:registro}
\centering
\begin{tabularx}{\linewidth}{@{}|c|X|c|@{}}
\hline
\rowcolor[HTML]{C0C0C0} 
Revisión & \multicolumn{1}{c|}{\cellcolor[HTML]{C0C0C0}Detalles de los cambios realizados} & Fecha      \\ \hline
1.0      & Creación del documento                                          & 25/08/2020 \\ \hline
1.1      & Se completaron los puntos del 1 al 6.                                                                 & 06/09/2020 \\ \hline
1.2      & Se aplicaron correcciones y se agregaron las historias de usuario. & 14/09/2020 \\ \hline
1.3      & Se aplicaron correcciones y se completaron los puntos de 7 al 11.                                                 & 21/09/2020 \\ \hline
\end{tabularx}
\end{table}

\pagebreak



\section{Acta de constitución del proyecto}
\label{sec:acta}

\begin{flushright}
Buenos Aires, \fechaInicioName
\end{flushright}

\vspace{2cm}

Por medio de la presente se acuerda con el Ing. \authorname\hspace{1px} que su Trabajo Final de la \degreename\hspace{1px} se titulará ``\ttitle'', consistirá esencialmente en el prototipo preliminar de un sistema de rastreo, seguimiento y control de ganado, y tendrá un presupuesto preliminar estimado de 790 hs de trabajo y \textcolor{red}{\$XXX}, con fecha de inicio \fechaInicioName\hspace{1px} y fecha de presentación pública \fechaFinalName.

Se adjunta a esta acta la planificación inicial.

\vfill

% Esta parte se construye sola con la información que hayan cargado en el preámbulo del documento y no debe modificarla
\begin{table}[ht]
\centering
\begin{tabular}{ccc}
\begin{tabular}[c]{@{}c@{}}Ariel Lutenberg \\ Director posgrado FIUBA\end{tabular} & \hspace{2cm} & \begin{tabular}[c]{@{}c@{}}\clientename \\ \empclientename \end{tabular} \vspace{2.5cm} \\ 
\multicolumn{3}{c}{\begin{tabular}[c]{@{}c@{}} \supname \\ Director del Trabajo Final\end{tabular}} \vspace{2.5cm} \\
%\begin{tabular}[c]{@{}c@{}}\jurunoname \\ Jurado del Trabajo Final\end{tabular}     &  & \begin{tabular}[c]{@{}c@{}}\jurdosname\\ Jurado del Trabajo Final\end{tabular}  \vspace{2.5cm}  \\
%\multicolumn{3}{c}{\begin{tabular}[c]{@{}c@{}} \jurtresname\\ Jurado del Trabajo Final\end{tabular}} \vspace{.5cm}                                                                     
\end{tabular}
\end{table}




\section{Descripción técnica-conceptual del proyecto a realizar}
\label{sec:descripcion}


La ganadería argentina está pasando por un cambio estructural con el objetivo de ajustarse a la demanda interna y recuperar al mismo tiempo protagonismo internacional. Los últimos años creció 10\% la producción de las cuatro carnes (vacuna, bovina, porcina y ave) hasta alcanzar los 6 millones de toneladas, de las cuales la carne vacuna representa el 50\% de la oferta. Actualmente la Argentina pasó a ocupar el 5to puesto en exportaciones de carne bovina, de las cuales el 75\% de los envíos fueron a China, lo que genera un saldo comercial total de 1.000 millones de dólares. Bajo este panorama, los avances tecnológicos innovadores se afianzan con el objetivo de lograr producciones más eficientes y controladas que contribuyan y disparen los beneficios del mercado. 

Bajo este escenario se espera implementar un sistema autónomo, de bajo costo y consumo que permita recopilar, almacenar, procesar y visualizar la posición de determinadas cabezas de ganado a través de GPS y que opere bajo una red LoRaWAN. Adicionalmente a esto, toda la información recolectada deberá ser almacenada y posteriormente procesada para ser visualizada en una aplicación que permita al agricultor/criador tener a disposición un historial de ubicaciones de sus animales en tiempo real. El sistema debe ser escalable, operar con batería y enviar periódicamente la posición de la cabeza de ganado.  En la figura 1. se puede observar la composición básica del sistema. 



\vspace{25px}

\begin{figure}[htpb]
\centering 
\includegraphics[width=.9\textwidth]{./Figuras/Diagrama.png}
\caption{Diagrama en bloques del sistema}
\label{fig:diagBloques}
\end{figure}

\vspace{25px}

El objetivo principal es implementar un pequeño dispositivo que pueda ser colocado en un collar lo suficientemente cómodo para el animal, resistente a condiciones ambientales extremas y que sea capaz de tomar la posición GPS con un sistema que comprenda un microcontrolador y hardware de bajo consumo con baterías recargables incorporando un cargador micro-USB. Al mismo tiempo deberá ser capaz de transmitir la posición y telemetría interna del dispositivo, con parámetros tales como ID, nivel de batería e intensidad de la señal haciendo uso de una red LoRaWAN que permita guardar los datos en la nube para su posterior procesamiento. 

El proyecto permitirá al agricultor/criador tener un total control de su ganado y le dará posibilidad de visualizar en tiempo real la ubicación de los mismos, disponer de un historial de movimiento y ubicación de cada animal, vista total de los limites de su terreno, deducir estado de pastizales en su territorio en base a un historial de movimiento de su ganado, deducir estado de salud de sus animales y control de robo o atentado; todos fuertes contribuyentes a aumentar la calidad, eficiencia y total control de su ganado. 


\section{Identificación y análisis de los interesados}
\label{sec:interesados}


\begin{table}[ht]
%\caption{Identificación de los interesados}
%\label{tab:interesados}
\begin{tabularx}{\linewidth}{@{}|l|X|X|l|@{}}
\hline
\rowcolor[HTML]{C0C0C0} 
Rol           & Nombre y Apellido & Organización 	& Puesto 	\\ \hline
%Auspiciante   &                   &              	&        	\\ \hline
Cliente       & \clientename      &\empclientename	& Gerente de Ingenieria       	\\ \hline
%Impulsor      &                   &              	&        	\\ \hline
Responsable   & \authorname       & FIUBA        	& Alumno 	\\ \hline
Colaboradores & Esp. Ing. Jhonattan Camargo                & Smartium             	& I+D       	\\ \hline
Orientador    & \supname	      & \pertesupname 	& Director	Trabajo final \\ \hline
%Equipo        & miembro1 \newline 
%				miembro2          &              	&        	\\ \hline
%Opositores    &                   &              	&        	\\ \hline
Usuario final & -                  & Criador que desee implementar una solución tecnológica en su ganado             	&  -      	\\ \hline
\end{tabularx}
\end{table}

El Mg. Ing. Osvaldo P. Ivani tiene muchos conocimientos técnicos y es un gran guia para la elaboración del proyecto pero no dispone de mucho tiempo por tener una gran responsabilidad laboral. 

El Esp. Ing. Jhonattan Camargo posee conocimientos del tema y será de gran ayuda para solventar dudas técnicas y del proyecto, también tiene una agenda ocupada y no dispone de mucho tiempo.



\section{1. Propósito del proyecto}
\label{sec:proposito}

El propósito de este proyecto es desarrollar el prototipo de un sistema autónomo capaz de transmitir la ubicación exacta de una cabeza de ganado y a partir de allí implementar una aplicación capaz de usar estos datos para relevar en tiempo real y de forma remota mas información al criador sobre el estado de su campo y su ganado. 

\section{2. Alcance del proyecto}
\label{sec:alcance}

El alcance de este proyecto incluirá un prototipo de hardware capaz de transmitir la posición a través de la red de comunicación LoRa y el desarrollo de una aplicación integrada con AWS cloud capaz de utilizar estos datos para mostrarlos en pantalla. Al mismo tiempo no contempla la conexión desde el Gateway hasta el servidor de Aplicación, la capacitación del usuario para hacer uso de la aplicación ni el diseño 3D del gabinete del collar. 


\section{3. Supuestos del proyecto}
\label{sec:supuestos}

Para el cumplimiento de este proyecto se supone que: 

\begin{itemize}
\item Se dispondrá del tiempo y espacio para realizar distintas pruebas.
\item Habrá recursos económicos para la construcción del hardware.
\item Los tiempos fabricación estarán dentro de los márgenes habituales.
\item Existirá comunicación permanente con el cliente para solventar dudas.
\item Los tiempos de autonomía del hardware estarán basados en cálculos pesimistas y no en ensayos controlados.
\end{itemize}


\section{4. Requerimientos}
\label{sec:requerimientos}

Los requerimientos del proyecto fueron convenidos por ambas partes a través del documento inicial de presentación del proyecto y distintas reuniones coordinadas. 


\begin{enumerate}
\item Requerimientos generales del proyecto:
	\begin{enumerate}
	\item Deadline 20 julio 2021
	\item Entregable prototipo de hardware funcional y prototipo de aplicación con función de visualización de terreno y ubicación en tiempo real. 
	\end{enumerate}
\item Requerimientos funcionales del sistema:
	\begin{enumerate}
	\item Hardware:
		\begin{enumerate}
		\item Hardware tipo collar con un peso no superior a los 250gr y protección IP46 o superior.
		\item Envío de coordenadas GPS a través de enlace de radiofrecuencia LoRaWAN en banda AU915.
		\item Bajo consumo con autonomía mínima de 1 año.
		\item Entrada de recarga de batería por micro-USB.
		%\item Bajo costo.
		\end{enumerate}
	\item Software y aplicación:
		\begin{enumerate}
		\item Almacenamiento de la información recolectada en la plataforma AWS Cloud.
		\item Validación de usuario con la plataforma Smartium.
		%\item Escalabilidad a nivel de software para nuevos collares.
		\item Creación y asociación de nuevos animales donde se podrán cargar identificador único, tipo, nombre, fecha, etc.
		\item Cálculos de distancia recorrida por el animal de forma diaria, cada tres días y semanales.
		\item Creación de zonas georeferenciadas para establecer si un animal está fuera del territorio.
		\item Visualización de la última posición de los animales asociados a un usuario junto con su respectivo historial de ubicación.
		\item Plataforma compatible con AWS Cloud.
		\item Bajo tráfico de datos.
		\end{enumerate}
	\end{enumerate}
\item Requerimientos de documentación:
	\begin{enumerate}
	\item Elaboración de un informe de avance del proyecto.
	\item Elaboración de informe de puesta en marcha.
	\end{enumerate}	
\end{enumerate}



\section{Historias de usuarios (\textit{Product backlog})}
\label{sec:backlog}

Ponderación: según el esfuerzo que requiera el cumplimiento de la historia.

Prioridad: de 1 al 10 considerando 10 como lo mas importante.

Se define usuario criador como el usuario final, dueño del campo.

Se define como usuario aquel quien hace un uso general del sistema.

Calificación: Ponderación - Prioridad.

\begin{itemize}
\item Como usuario criador quiero una aplicación que me permita visualizar el campo. \textit{7-10}
\item Como gerente de Smartium quiero un dispositivo que trasmita su posición para geolocalizarla posteriormente. \textit{10-10}
\item Como gerente de Smartium quiero que el dispositivo sea compatible con la infraestructura ya existente para mayor interoperabilidad y soporte. \textit{7-10}
\item Como gerente de Smartium quiero que la información sea visible en una aplicación para tener mayor portabilidad. \textit{10-10}
\item Como gerente de Smartium quiero una interfaz gráfica para tener un login de usuario. \textit{7-10}
\item Como usuario criador quiero una aplicación para que funcione en un smartphone. \textit{7-7}
\item Como usuario criador quiero una aplicación para poder ver donde se mueve el ganado en distintos intervalos de tiempo. \textit{7-7}
\item Como gerente de Smartium quiero un dispositivo funcione al menos 1 año para facilitar la instalación y el mantenimiento. \textit{7-10}
\item Como gerente de Smartium quiero un sistema compatible con AWS Cloud para hacer uso de la infraestructura existente. \textit{7-10}
\item Como gerente de Smartium quiero un dispositivo que sea de bajo consumo para facilitar su uso y mantenimiento. \textit{7-10}
\item Como gerente de Smartium quiero que el sistema sea escalable para permitir agregar nuevos dispositivos. \textit{10-10}
\item Como usuario criador quiero acceder a una web para ver el ganado. \textit{10-10}
\item Como gerente de Smartium quiero un dispositivo que sea recargable para fomentar su re-utilidad. \textit{5-10}
\item Como gerente de Smaritum quiero un sistema de baja tasa de transmisión de datos para ampliar su utilidad en áreas sub-urbanas. \textit{10-10}
\item Como usuario criador quiero un dispositivo que sea ligero para no perjudicar al animal. \textit{5-10}
\item Como usuario criador quiero un dispositivo que sea sencillo de colocar para facilitar su uso. \textit{5-10}
\item Como usuario criador quiero un dispositivo que no afecte el comportamiento del animal para cuidar su integridad y desarrollo. \textit{5-10}
\item Como gerente de Smartium quiero una aplicación multi-plataforma para no limitar su uso a ciertos dispositivos. \textit{10-10}
\item Como usuario criador quiero un sistema que permita relevar el comportamiento del animal para determinar una posible enfermedad. \textit{10-7}
\item Como usuario criador quiero visualizar un mapa topográfico del terreno para conocer el estado de los pastizales. \textit{10-7}
\item Como usuario criador quiero una interfaz gráfica para poder nombrar e identificar un animal en particular. \textit{7-7}   
\end{itemize}

\section{5. Entregables principales del proyecto}
\label{sec:entregables}


\begin{itemize}
\item Planificación del trabajo final.
\item Prototipo funcional del hardware y software de aplicación.
\item Código fuente.
\item Diagrama esquemático del hardware.
\item Informe final.

\end{itemize}

\section{6. Desglose del trabajo en tareas}
\label{sec:wbs}


\begin{enumerate}
\item Análisis inicial (130)
	\begin{enumerate}
	\item Lectura de referencias bibliográficas. (60)
	\item Análisis de las herramientas trabajo. (30)
	\item Elaboración de la planificación del proyecto. (40)
	\end{enumerate}
\item Diseño de Hardware (135)
	\begin{enumerate}
	\item Diseño estructura del sistema. (40)
	\item Diseño preliminar del circuito. (15)
	\item Selección de componentes. (15)
	\item Elaboración del circuito esquemático. (15)
	\item Análisis preliminar de consumo. (10)
	\item Diseño de PCB. (20)
	\item Dimensionado del prototipo físico. (20)
	\end{enumerate}
\item Desarrollo de Firmware (155)
	\begin{enumerate}
	\item Configuración de entorno de diseño. (15)
	\item Diseño de la estructura del firmware. (15)
	\item Selección de interfaz de comunicación. (15)
	\item Implementación de adquisición de datos. (40)
	\item Trama de comunicación. (40)
	\item Configuración de la estrategia de control del sistema. (30)
	\end{enumerate}
\item Producción (60)
	\begin{enumerate}
	\item Fabricación del PCB y montaje de componentes. (40)
	\item Montaje final del prototipo. (20)
	\end{enumerate}
\item Testing (50)
	\begin{enumerate}
	\item Comprobación de funcionamiento del circuito. (10)
	\item Ensayo y medición de consumo. (15)
	\item Ensayo de comunicación con la red LoRaWAN. (15)
	\item Checkeo de la correcta llegada de datos al Cloud. (10)
	\end{enumerate}
\item Aplicación (130)
	\begin{enumerate}
	\item Selección de estrategia y entorno de programación. (20)
	\item Comunicación y permisos con la nube. (20)
	\item Desarrollo del login de usuario. (10)
	\item Comprobación de comunicación y login correctos. (10)
    \item Desarrollo de interfaz de usuario. (40)
	\item Desarrollo de algoritmos de historial de ubicación. (30)
	\end{enumerate}
\item Ensayos sobre aplicación (20)
	\begin{enumerate}
	\item Pruebas y ensayos sobre la aplicación. (10)
	\item Corrección de errores y ajustes menores. (10)
	\end{enumerate}
\item Cierre del proyecto (110)
	\begin{enumerate}
	\item Elaboración de documentación. (20)
	\item Elaboración de la memoria técnica del trabajo final. (60) 
	\item Elaboración de la presentación del proyecto final. (20)
	\item Cierre del proyecto. (10)
	\end{enumerate}
\end{enumerate}

Cantidad total de horas: (790)
 


\section{7. Diagrama de Activity On Node}
\label{sec:AoN}

Los tiempos del diagrama de la Fig.2 están en horas con excepción de las fechas de inicio y fin.

\begin{figure}[htpb]
\centering 
\includegraphics[width=1\textwidth]{./Figuras/AoNb3.png}
\caption{Diagrama en \textit{Activity on Node}}
\label{fig:AoN}
\end{figure}

La tarea 4.1 tiene riesgo de sufrir retrasos de tal manera de establecerse como un camino crítico. 

\section{8. Diagrama de Gantt}
\label{sec:gantt}

\begin{figure}[htpb]
\centering 
\includegraphics[width=.88\textwidth]{./Figuras/GanttIoT_tabla.png}
\caption{Diagrama en \textit{Tabla de WBS}}
\label{fig:AoN}
\end{figure}

\begin{figure}[htpb]
\centering 
\includegraphics[angle=90, width=.91\textwidth]{./Figuras/GanttIoT_figura.png}
\caption{Diagrama en \textit{Diagrama de Gantt}}
\label{fig:AoN}
\end{figure}


\section{9. Matriz de uso de recursos de materiales}
\label{sec:recursos}

%\begin{table}
%\label{tab:recursos}
%\centering
%\begin{tabularx}{\linewidth}{@{}|c|X|X|X|X|X|X|c|@{}}
%\hline
%\cellcolor[HTML]{C0C0C0} & \cellcolor[HTML]{C0C0C0} & \multicolumn{6}{c|}{\cellcolor[HTML]{C0C0C0}Recursos requeridos (expresado en horas)} \\ \cline{4-6} 
%\multirow{-2}{*}{\cellcolor[HTML]{C0C0C0}\begin{tabular}[c]{@{}c@{}}Código\\ WBS\end{tabular}} & \multirow{-2}{*}{\cellcolor[HTML]{C0C0C0}\begin{tabular}[c]{@{}c@{}}Nombre \\ tarea\end{tabular}} & PC & xDot-DK & Prototipo & Gateway & AWS Cloud & Workspace\\ \hline
% 1.&  Análisis inicial&  130&  -&  -&  -& -& \\ \hline
% 2.4.& Elaboración del circuito esquemático &  &  &  &  & & \\ \hline
% 2.6.& Diseño de PCB &  &  &  &  & &\\ \hline
% 2.8.& Elaboración de protocolos de transmisión &  &  &  &  & &\\ \hline
% 3.& Producción &  &  &  &  & &\\ \hline 
% 4.& Testing &  &  &  &  & &\\ \hline
% 5.2& Comunicación y permisos con la nube &  &  &  &  & &\\ \hline
% 5.3.& Desarrollo del login de usuario &  &  &  &  & &\\ \hline
% 5.4.& Comprobación de comunicación y login correctos  &  &  &  &  & &\\ \hline
% 5.5.& Desarrollo de interfaz de usuario &  &  &  &  & &\\ \hline
% 5.6.& Desarrollo de algoritmos de historial de ubicación &  &  &  &  & &\\ \hline
% 6.& Ensayos sobre la aplicación &  &  &  &  & &\\ \hline
% 7.& Cierre de proyecto &  &  &  &  & &\\ \hline
%\end{tabularx}%
%\end{table}


\begin{table}[h]
\label{tab:recursos}
%\begin{tabular}{\linewidth}{@{}|c|X|X|X|X|X|X|c|@{}}
\begin{tabular}{|l|c|l|l|l|l|l|l|}
\hline
\multicolumn{1}{|c|}{\multirow{2}{*}{\begin{tabular}[c]{@{}c@{}}Código\\  WBS\end{tabular}}} &
  \multirow{2}{*}{Nombre de la tarea} &
  \multicolumn{6}{c|}{Recursos requeridos (en horas)} \\ \cline{3-8} 
\multicolumn{1}{|c|}{} &
   &
  \multicolumn{1}{c|}{PC} &
  \multicolumn{1}{c|}{xDOT-DK} &
  \multicolumn{1}{c|}{\begin{tabular}[c]{@{}c@{}}Altium \\ Desing\end{tabular}} &
  \multicolumn{1}{c|}{Gateway} &
  \multicolumn{1}{c|}{\begin{tabular}[c]{@{}c@{}}AWS \\ Cloud\end{tabular}} &
  Workspace \\ \hline
1.   & Análisis inicial                                                                                 &130  &-  &-  &-  &-  &-  \\ \hline
2.4. & \begin{tabular}[c]{@{}c@{}}Elaboración del circuito \\ esquemático\end{tabular}                  &15  &-  &15  &-  &-  &-  \\ \hline
2.6. & Diseño de PCB                                                                                    &20  &-  &20  &-  &-  &-  \\ \hline
3.1 & \begin{tabular}[c]{@{}c@{}}Configuración del \\  entorno de diseño\end{tabular}              &15  &15  &-  &-  &-  &-  \\ \hline
3.2 & \begin{tabular}[c]{@{}c@{}}Diseño de la  \\  estructura del \\ firmware \end{tabular}              &15  &15  &-  &-  &-  &-  \\ \hline
3.3 & \begin{tabular}[c]{@{}c@{}}Selección interfaz \\  de comunicación\end{tabular}              &15  &15  &-  &-  &-  &-  \\ \hline
3.4 & \begin{tabular}[c]{@{}c@{}}Implementación de \\  adquisición\end{tabular}              &40  &40  &-  &20  &-  &-  \\ \hline
3.5 & \begin{tabular}[c]{@{}c@{}}Trama de \\ comunicación \end{tabular}              &40  &40  &-  &20  &-  &-  \\ \hline
3.6 & \begin{tabular}[c]{@{}c@{}}Configuración de \\  la estrategia\\ general de \\ control \end{tabular}              &30  &30  &-  &15  &-  &-  \\ \hline
4.   & Producción                                                                                       &-  &-  &-  &-  &-  &60  \\ \hline
5.   & Testing                                                                                          &30  &-  &-  &50  &50  &20  \\ \hline
6.2. & \begin{tabular}[c]{@{}c@{}}Comunicación y permisos \\ con la nube\end{tabular}                   &20  &-  &-  &-  &20  &-  \\ \hline
6.3. & \begin{tabular}[c]{@{}c@{}}Desarrollo del login \\ de usuario\end{tabular}                       &10  &-  &-  &-  &10  &-  \\ \hline
6.4. & \begin{tabular}[c]{@{}c@{}}Comprobación de \\ comunicación y \\ login correctos\end{tabular}     &10  &-  &-  &-  &10  &-  \\ \hline
6.5. & \begin{tabular}[c]{@{}c@{}}Desarrollo de interfaz \\ de usuario\end{tabular}                     &40  &-  &-  &-  &40  &-  \\ \hline
6.6. & \begin{tabular}[c]{@{}c@{}}Desarrollo de \\ algoritmos de historial \\ de ubicación\end{tabular} &30  &-  &-  &-  &30  &-  \\ \hline
7.   & \begin{tabular}[c]{@{}c@{}}Ensayo sobre \\ la aplicación\end{tabular}                            &20  &-  &-  &-  &20  &-  \\ \hline
8.   & Cierre de proyecto                                                                               &110  &-  &-  &-  &-  &110  \\ \hline
\end{tabular}
\end{table}

\vspace{260px}

\section{10. Presupuesto detallado del proyecto}
\label{sec:presupuesto}

El presupuesto corresponde a un estimado en dólares, considerando un costo por hora de ingeniería de AR\$350 (u\$d4.64) y un aproximado de costos indirectos del 30\% de los costos directos. No fue incluido ni considerado el costo del Gateway ni del servicio de AWS Cloud. 

\vspace{10px}

\begin{table}[htpb]
\label{tab:presupuesto}
\begin{tabular}{|c|c|c|c|c|}
\hline
Categoría & Detalle                  & Cantidad & Valor unitario (Dólares) & Costo (Dólares) \\ \hline
          & Horas de Ingeniería      & 790      & 4.64                     & 3665.6          \\ \cline{2-5} 
          & Módulo xDOT              & 1        & 7.3                      & 7.3             \\ \cline{2-5} 
          & Módulo GPS               & 1        & 38                       & 38              \\ \cline{2-5} 
          & Módulo Regulador         & 1        & 18                       & 18              \\ \cline{2-5} 
          & Componentes electrónicos & -        & 38                       & 38              \\ \cline{2-5} 
          & Fabricación de PCB       & 1        & 50                       & 50              \\ \cline{2-5} 
          & Fabricación de gabinete  & 1        & 33                       & 33              \\ \cline{2-5} 
\multirow{-8}{*}{\begin{tabular}[c]{@{}c@{}}Costos\\ Directos\end{tabular}}    & Montaje del prototipo         & -         & 10        & 10                             \\ \hline
\rowcolor[HTML]{FFFFC7} 
\multicolumn{4}{|c|}{\cellcolor[HTML]{FFFFC7}Subtotal}                     & 3859.9          \\ \hline
          & \multicolumn{3}{c|}{30\% de los costos directos}               & 1157.97          \\ \cline{2-5} 
\multirow{-2}{*}{\begin{tabular}[c]{@{}c@{}}Costos \\ Indirectos\end{tabular}} & \multicolumn{3}{c|}{\cellcolor[HTML]{FFFFC7}Subtotal} & \cellcolor[HTML]{FFFFC7}938.97 \\ \hline
\rowcolor[HTML]{FFCE93} 
\multicolumn{4}{|c|}{\cellcolor[HTML]{FFCE93}Total}                        & 5017.87         \\ \hline
\end{tabular}
\end{table}

\vspace{400px}

\section{11. Matriz de asignación de responsabilidades}
\label{sec:responsabilidades}

\begin{table}[h]
\begin{tabular}{|c|c|c|c|c|c|}
\hline
\multirow{2}{*}{\begin{tabular}[c]{@{}c@{}}Código\\ WBS\end{tabular}} &
  \multirow{2}{*}{\begin{tabular}[c]{@{}c@{}}Nombre de \\ la tarea\end{tabular}} &
  Responsable &
  Orientador &
  Colaboradores &
  Cliente \\ \cline{3-6} 
 &
   &
  \begin{tabular}[c]{@{}c@{}}Christian \\ Canaan \\ Castro Botek\end{tabular} &
  \begin{tabular}[c]{@{}c@{}}Nombre del \\ Director\end{tabular} &
  \begin{tabular}[c]{@{}c@{}}Esp. Ing. \\ Jhonattan\\  Camargo\end{tabular} &
  \begin{tabular}[c]{@{}c@{}}Mg. Ing. \\ Osvaldo \\ P. Ivani\end{tabular} \\ \hline
1.1. & \begin{tabular}[c]{@{}c@{}}Lectura de Ref. \\ Bibliográficas\end{tabular}                 & P &   &   &   \\ \hline
1.2. & \begin{tabular}[c]{@{}c@{}}Análisis de las \\ herramientas \\ de trabajo\end{tabular}     & P &   & C & C \\ \hline
1.3. & \begin{tabular}[c]{@{}c@{}}Elaboración de \\ la planificación \\ de proyecto\end{tabular} & P &   &   &   \\ \hline
2.1. & \begin{tabular}[c]{@{}c@{}}Diseño de la \\ estructura del \\ sistema\end{tabular}         & P &   & C & C \\ \hline
2.2. & \begin{tabular}[c]{@{}c@{}}Diseño preliminar \\ del circuito\end{tabular}                 & P & C & C & C \\ \hline
2.3. & \begin{tabular}[c]{@{}c@{}}Selección de \\ componentes\end{tabular}                       & P &   & A & A \\ \hline
2.4. & \begin{tabular}[c]{@{}c@{}}Elaboración del \\ circuito \\ esquemático\end{tabular}        & P & C &   &   \\ \hline
2.5. & \begin{tabular}[c]{@{}c@{}}Análisis preliminar \\ de consumo\end{tabular}                 & P & C &   &   \\ \hline
2.6. & Diseño de PCB                                                                             & P &  & A & A \\ \hline
2.7. & \begin{tabular}[c]{@{}c@{}}Dimensionado del \\ prototipo físico\end{tabular}              & P &   & A & A \\ \hline
3.   & \begin{tabular}[c]{@{}c@{}}Desarrollo de \\ firmware\end{tabular}                         & P & C & C & C \\ \hline
4.   & Producción                                                                                & P & C & I & I \\ \hline
5.   & Testing                                                                                   & P & C & I & I \\ \hline
6.   & Aplicación                                                                                & P & C & C & C \\ \hline
7.   & \begin{tabular}[c]{@{}c@{}}Ensayo sobre la \\ aplicación\end{tabular}                     & P &   & I & I \\ \hline
8.   & Cierre de proyecto                                                                        & P & A & I & A \\ \hline
\end{tabular}
\end{table}

Referencias:
\begin{itemize}
	\item P = Responsabilidad Primaria
	\item S = Responsabilidad Secundaria
	\item A = Aprobación
	\item I = Informado
	\item C = Consultado
\end{itemize}



\section{12. Gestión de riesgos}
\label{sec:riesgos}

\begin{consigna}{red}
a) Identificación de los riesgos (al menos cinco) y estimación de sus consecuencias:
 
Riesgo 1: detallar el riesgo (riesgo es algo que si ocurre altera los planes previstos)
\begin{itemize}
\item Severidad (S): mientras más severo, más alto es el número (usar números del 1 al 10).\\
Justificar el motivo por el cual se asigna determinado número de severidad (S).
\item Probabilidad de ocurrencia (O): mientras más probable, más alto es el número (usar del 1 al 10).\\
Justificar el motivo por el cual se asigna determinado número de (O). 
\end{itemize}   

Riesgo 2:
\begin{itemize}
\item Severidad (S): 
\item Ocurrencia (O):
\end{itemize}

Riesgo 3:
\begin{itemize}
\item Severidad (S): 
\item Ocurrencia (O):
\end{itemize}


b) Tabla de gestión de riesgos:      (El RPN se calcula como RPN=SxO)

\begin{table}[htpb]
\centering
\begin{tabularx}{\linewidth}{@{}|X|c|c|c|c|c|c|@{}}
\hline
\rowcolor[HTML]{C0C0C0} 
Riesgo & S & O & RPN & S* & O* & RPN* \\ \hline
       &   &   &     &    &    &      \\ \hline
       &   &   &     &    &    &      \\ \hline
       &   &   &     &    &    &      \\ \hline
       &   &   &     &    &    &      \\ \hline
       &   &   &     &    &    &      \\ \hline
\end{tabularx}%
\end{table}

Criterio adoptado: 
Se tomarán medidas de mitigación en los riesgos cuyos números de RPN sean mayores a...

Nota: los valores marcados con (*) en la tabla corresponden luego de haber aplicado la mitigación.

c) Plan de mitigación de los riesgos que originalmente excedían el RPN máximo establecido:
 
Riesgo 1: plan de mitigación (si por el RPN fuera necesario elaborar un plan de mitigación).
  Nueva asignación de S y O, con su respectiva justificación:
  - Severidad (S): mientras más severo, más alto es el número (usar números del 1 al 10).
          Justificar el motivo por el cual se asigna determinado número de severidad (S).
  - Probabilidad de ocurrencia (O): mientras más probable, más alto es el número (usar del 1 al 10).
          Justificar el motivo por el cual se asigna determinado número de (O).

Riesgo 2: plan de mitigación (si por el RPN fuera necesario elaborar un plan de mitigación).
 
Riesgo 3: plan de mitigación (si por el RPN fuera necesario elaborar un plan de mitigación).

\end{consigna}


\section{13. Gestión de la calidad}
\label{sec:calidad}

\begin{consigna}{red}
Para cada uno de los requerimientos del proyecto indique:
\begin{itemize} 
\item Req \#1: copiar acá el requerimiento.

Verificación y validación:

\begin{itemize}
\item Verificación para confirmar si se cumplió con lo requerido antes de mostrar el sistema al cliente. Detallar 
\item Validación con el cliente para confirmar que está de acuerdo en que se cumplió con lo requerido. Detallar  
\end{itemize}

\end{itemize}

Tener en cuenta que en este contexto se pueden mencionar simulaciones, cálculos, revisión de hojas de datos, consulta con expertos, mediciones, etc.

\end{consigna}

\section{14. Comunicación del proyecto}
\label{sec:comunicaciones}

El plan de comunicación del proyecto es el siguiente:

\begin{table}[htpb]
\centering
\begin{tabularx}{\linewidth}{@{}|X|C{2.4cm}|C{3cm}|C{1.8cm}|C{2cm}|C{2.1cm}|@{}}
\hline
\rowcolor[HTML]{C0C0C0} 
\multicolumn{6}{|c|}{\cellcolor[HTML]{C0C0C0}PLAN DE COMUNICACIÓN DEL PROYECTO}           \\ \hline
\rowcolor[HTML]{C0C0C0} 
¿Qué comunicar? & Audiencia & Propósito & Frecuencia & Método de comunicac. & Responsable \\ \hline
                &           &           &            &                      &             \\ \hline
                &           &           &            &                      &             \\ \hline
                &           &           &            &                      &             \\ \hline
                &           &           &            &                      &             \\ \hline
                &           &           &            &                      &             \\ \hline
\end{tabularx}
\end{table}

\section{15. Gestión de compras}
\label{sec:compras}

\begin{consigna}{red}
En caso de tener que comprar elementos o contratar servicios:
a) Explique con qué criterios elegiría a un proveedor.
b) Redacte el Statement of Work correspondiente.
\end{consigna}

\section{16. Seguimiento y control}
\label{sec:seguimiento}

\begin{consigna}{red}
Para cada tarea del proyecto establecer la frecuencia y los indicadores con los se seguirá su avance y quién será el responsable de hacer dicho seguimiento y a quién debe comunicarse la situación (en concordancia con el Plan de Comunicación del proyecto).

El indicador de avance tiene que ser algo medible, mejor incluso si se puede medir en \% de avance. Por ejemplo,se pueden indicar en esta columna cosas como ``cantidad de conexiones ruteadeas'' o ``cantidad de funciones implementadas'', pero no algo genérico y ambiguo como ``\%'', porque el lector no sabe porcentaje de qué cosa.

\end{consigna}

\begin{longtable}{|m{1cm}|m{3.5cm}|m{2.2cm}|m{2cm}|m{3cm}|m{1.5cm}|}
\hline
\rowcolor[HTML]{C0C0C0} 
\multicolumn{6}{|c|}{\cellcolor[HTML]{C0C0C0}SEGUIMIENTO DE AVANCE}                                                                       \\ \hline
\rowcolor[HTML]{C0C0C0} 
Tarea del WBS 			& Indicador de avance & Frecuencia de reporte & Resp. de seguimiento & Persona a ser informada & Método de comunic. \\ \hline
\endfirsthead

\hline
\rowcolor[HTML]{C0C0C0} 
\multicolumn{6}{c}{\cellcolor[HTML]{C0C0C0}SEGUIMIENTO DE AVANCE}                                                                       \\ \hline
\rowcolor[HTML]{C0C0C0} 
Tarea del WBS 			& Indicador de avance & Frecuencia de reporte & Resp. de seguimiento & Persona a ser informada & Método de comunic. \\ \hline
\endhead

\multicolumn{6}{c}{Continúa}
\endfoot

\endlastfoot

1.1	& Fecha de inicio  & Única vez al comienzo & \authorname & \clientename, \supname & email \\ \hline
2.1	& Avance de las subtareas  & Mensual mientras dure la tarea & \authorname & \clientename, \supname & email \\ \hline

\end{longtable}

\begin{table}[!htpb]
\centering
%\begin{tabularx}{\linewidth}{@{}|X|X|X|X|X|X|@{}}
\begin{tabularx}{\linewidth}{@{}|X|C{2.5cm}|C{3cm}|C{2cm}|C{2cm}|C{2.5cm}|@{}}
\hline
\rowcolor[HTML]{C0C0C0} 
\multicolumn{6}{|c|}{\cellcolor[HTML]{C0C0C0}SEGUIMIENTO DE AVANCE}                                                                       \\ \hline
\rowcolor[HTML]{C0C0C0} 
Tarea del WBS & Indicador de avance & Frecuencia de reporte & Resp. de seguimiento & Persona a ser informada & Método de comunic. \\ \hline
 &  &  &  &  &  \\ \hline
 &  &  &  &  &  \\ \hline
 &  &  &  &  &  \\ \hline
 &  &  &  &  &  \\ \hline
 &  &  &  &  &  \\ \hline
\end{tabularx}%
%}
\end{table}

\section{17. Procesos de cierre}    
\label{sec:cierre}

\begin{consigna}{red}
Establecer las pautas de trabajo para realizar una reunión final de evaluación del proyecto, tal que contemple las siguientes actividades:

\begin{itemize}
\item Pautas de trabajo que se seguirán para analizar si se respetó el Plan de Proyecto original:
 - Indicar quién se ocupará de hacer esto y cuál será el procedimiento a aplicar. 
\item Identificación de las técnicas y procedimientos útiles e inútiles que se utilizaron, y los problemas que surgieron y cómo se solucionaron:
 - Indicar quién se ocupará de hacer esto y cuál será el procedimiento para dejar registro.
\item Indicar quién organizará el acto de agradecimiento a todos los interesados, y en especial al equipo de trabajo y colaboradores:
  - Indicar esto y quién financiará los gastos correspondientes.
\end{itemize}

\end{consigna}


\end{document}
